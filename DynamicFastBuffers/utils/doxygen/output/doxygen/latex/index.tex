\begin{center}{\bfseries D\-Y\-N\-A\-M\-I\-C F\-A\-S\-T B\-U\-F\-F\-E\-R\-S Library}\end{center} \par




\par
\begin{center}\href{http://www.eprosima.com}{\tt e\-Prosima}\end{center} 

e\-Prosima Dynamic Fast Buffers (D\-F\-B) is a high-\/performance library that allows users to describe and serialize or deserialize data dynamically in run-\/time. Its functionality is based on the creation of a typecode for data definition, and generate afterwards a bytecode to do data serialization. There is no need for users to know any internal data of the typecode of the bytecode, neither of the serialization procedure.

This library uses e\-Prosima Fast Buffers for data serialization and deserialization, for in it are defined the functions that perform this operations.

e\-Prosima D\-F\-B also brings this features\-:

\begin{DoxyItemize}
\item Data description through a typecode\-: The typecode is a way to describe how is the data that any user wants to use in its application. Through this typecode, e\-Prosima D\-F\-B knows how is the user's data and how to move along it. Avoids the developer to describe data statically inside an I\-D\-L file. \item Bytecode generation for data serialization/deserialization\-: This is just a way for D\-F\-B of knowing how to serialize the data defined by user. There are two kinds of bytecode that can be generated, one specific for data serialization, and the other for data deserialization. \item Data serialization/deserialization\-: e\-Prosima D\-B\-F provides users a way to serialize or deserialize data by using a Fast\-C\-D\-R object provided by e\-Prosima Fast Buffers library. Serialized data will be stored inside a buffer defined by user. Deserialized data will be stored in the user's data type previously defined. \end{DoxyItemize}
